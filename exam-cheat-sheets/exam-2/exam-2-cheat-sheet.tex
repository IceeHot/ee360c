\documentclass[8pt]{article}

\usepackage{mathtools}
\usepackage{amsfonts}
\usepackage{tabularx}
\usepackage{mathabx}
\usepackage{enumitem}
\usepackage{caption}
\usepackage{multicol}
\usepackage{sectsty}
\usepackage{extsizes}
\usepackage{fancyhdr}
\usepackage{enumerate}
\usepackage[margin=1.5cm]{geometry}
\usepackage{multirow}

\pagestyle{fancyplain}

% Try very hard not to break relational and binary operators (equations)
\relpenalty=9999
\binoppenalty=9999

\newcommand{\dd}[1]{\mathrm{d}{#1}}
\newcommand{\ddt}[1]{\frac{\dd{}}{\dd{#1}}}
\newcommand{\dddt}[1]{\frac{\dd{}^2}{\dd{#1}^2}}

\begin{document}

\lhead{Hershal Bhave \quad\tiny{hb6279}}
\rhead{EE360C Fall 2014 Exam 2 Cheat Sheet (Pedro Santacruz)}

\begin{multicols}{2}
  \begin{description}
  \item[Adjacency List] Given a graph $G=(V,E)$, $\forall u \in V$,
    \verb|Adj[u]| is a list of vertices $v$ such that there is an edge
    from $u$ to $v$. Alternatively, using an $n \times n$ upper
    triangular matrix representation we can represent the graph and
    its edges in a space-effecient format $\mathbb{O}(n^2)$ memory
    locations where $a_{i,j}=\verb|weight|$.
  \item[BFS] Applications
    \begin{itemize}
    \item Finding Shortest Path (is optimal)
    \item Testing for bipartiteness
    \item Serialization
    \end{itemize}
    { \footnotesize
\begin{verbatim}
procedure BFS(G,v) is
    create a queue Q
    create a set V
    create a map P
    add v to V
    enqueue v onto Q
    while Q is not empty loop
       t ← Q.dequeue()
       if t is what we are looking for then
          return t
       end if
       for all edges e in G.adjacentEdges(t) loop
          u ← G.adjacentVertex(t,e)
          if u is not in V then
              add u to V
              enqueue u onto Q
              P[u] = t;
          end if
       end loop
    end loop
    return none
end BFS
\end{verbatim}
      Then backtrace through \verb|P| to obtain the path
    }
  \item[DFS]
    { \footnotesize
\begin{verbatim}
auto undirected_graph::dfs(graph_query* query,
                           size_t curr_dev_id,
                           size_t curr_edge_weight) -> bool {

    if (curr_dev_id == query->get_device_j()) {
        /* Found */
        return true;
    } else {
        auto adj_nodes = nodes[curr_dev_id]->get_adjacent_nodes();

        for (const auto an : adj_nodes) {
            if ((an->nref->is_enabled(curr_dev_id)) &&
                (curr_edge_weight >= query->get_time_a()) &&
                (an->edge_weight >= curr_edge_weight) &&
                (an->edge_weight <= query->get_time_b())) {

                /* Disable the connection */
                an->nref->disable(curr_dev_id);
                nodes[curr_dev_id]->disable(an->nref->get_id());

                query->push_trace(curr_dev_id, an->nref->get_id(), an->edge_weight);
                if (dfs(query, an->nref->get_id(), an->edge_weight)) {
                    return true;
                } else {
                    query->pop_trace();
                }
                /* Not found, then backtrack */
            }
        }
    }
    return false;
}
\end{verbatim}
    }
  \item[Bipartite Graphs] An undirected $G=(V,E)$ is bipartite if the
    nodes can be two-colored such that every edge is a different
    color. If $G$ is bipartite it cannot contain an odd length
    cycle. Using BFS at node $s$, only one of the following holds:
    \begin{itemize}
    \item No edge of $G$ joins two nodes of the same layer
    \item An edge of $G$ joins two nodes in the same layer and $G$
      contains an odd length cycle (not bipartite)
    \end{itemize}
  \item[Test for Bipartiteness] DFS through the tree. If there is any
    edge which connects to any node in the same layer, then it is not
    two-colorable/bipartite and there exists an odd cycle. Otherwise
    is two-colorable/bipartite.
  \end{description} % multicols
\end{multicols}
\end{document}