\documentclass[8pt]{article}

\usepackage{mathtools}
\usepackage{amsfonts}
\usepackage{tabularx}
\usepackage{mathabx}
\usepackage{enumitem}
\usepackage{caption}
\usepackage{multicol}
\usepackage{sectsty}
\usepackage{extsizes}
\usepackage{fancyhdr}
\usepackage{enumerate}
\usepackage[margin=1.5cm]{geometry}
\usepackage{multirow}

\pagestyle{fancyplain}

% Try very hard not to break relational and binary operators (equations)
\relpenalty=9999
\binoppenalty=9999

\newcommand{\dd}[1]{\mathrm{d}{#1}}
\newcommand{\ddt}[1]{\frac{\dd{}}{\dd{#1}}}
\newcommand{\dddt}[1]{\frac{\dd{}^2}{\dd{#1}^2}}

\begin{document}

\lhead{Hershal Bhave \quad\tiny{hb6279}}
\rhead{EE360C Fall 2014 Exam 2 Cheat Sheet (Pedro Santacruz)}

\begin{multicols}{2}
  \begin{description}
  \item[Adjacency List] Given a graph $G=(V,E)$, $\forall u \in V$,
    \verb|Adj[u]| is a list of vertices $v$ such that there is an edge
    from $u$ to $v$. Alternatively, using an $n \times n$ upper
    triangular matrix representation we can represent the graph and
    its edges in a space-effecient format $\mathcal{O}(n^2)$ memory
    locations where $a_{i,j}=\verb|weight|$.
  \item[BFS] Applications in finding shortest path, testing for
    bipartiteness, and serialization
    { \footnotesize
\begin{verbatim}
procedure BFS(G,v) is
    create a queue Q
    create a set V
    create a map P
    add v to V
    enqueue v onto Q
    while Q is not empty loop
       t <- Q.dequeue()
       if t is what we are looking for then
          return t
       end if
       for all edges e in G.adjacentEdges(t) loop
          u ← G.adjacentVertex(t,e)
          if u is not in V then
              add u to V
              enqueue u onto Q
              P[u] = t;
          end if
       end loop
    end loop
    return none
end BFS
\end{verbatim}
      Then backtrace through \verb|P| to obtain the path
    }
  \item[DFS]
    { \footnotesize
\begin{verbatim}
auto undirected_graph::dfs(graph_query* query,
                           size_t curr_dev_id,
                           size_t curr_edge_weight) -> bool {

    if (curr_dev_id == query->get_device_j()) {
        /* Found */
        return true;
    } else {
        auto adj_nodes = nodes[curr_dev_id]->get_adjacent_nodes();

        for (const auto an : adj_nodes) {
            if ((an->nref->is_enabled(curr_dev_id)) &&
                (curr_edge_weight >= query->get_time_a()) &&
                (an->edge_weight >= curr_edge_weight) &&
                (an->edge_weight <= query->get_time_b())) {

                /* Disable the connection */
                an->nref->disable(curr_dev_id);
                nodes[curr_dev_id]->disable(an->nref->get_id());

                query->push_trace(curr_dev_id, an->nref->get_id(), an->edge_weight);
                if (dfs(query, an->nref->get_id(), an->edge_weight)) {
                    return true;
                } else {
                    query->pop_trace();
                }
                /* Not found, then backtrack */
            }
        }
    }
    return false;
}
\end{verbatim}
    }
  \item[Bipartite Graphs] An undirected $G=(V,E)$ is bipartite if the
    nodes can be two-colored such that every edge is a different
    color. If $G$ is bipartite it cannot contain an odd length
    cycle. Using BFS at node $s$, only one of the following holds:
    \begin{itemize}
    \item No edge of $G$ joins two nodes of the same layer
    \item An edge of $G$ joins two nodes in the same layer and $G$
      contains an odd length cycle (not bipartite)
    \end{itemize}
  \item[Test for Bipartiteness] DFS through the tree. If there is any
    edge which connects to any node in the same layer, then it is not
    two-colorable/bipartite and there exists an odd cycle. Otherwise
    is two-colorable/bipartite.
  \item[Determine Strongly Connected] We can decide if $G$ is strongly
    connected in $\mathcal{O}(m+n)$ time.
    { \footnotesize
\begin{verbatim}
Pick any node s
bfs(s)
reverse_digraph(G)
bfs(s)
return (all nodes reached in both bfs)
\end{verbatim}
    }
  \item[DAG] Directed Acyclic Graph, has precedence constraint such that
    for an edge $(v_i,v_j)$, $v_i$ must come before $v_j$. Dags also
    contain a node which has no incoming edges.
  \item[Topological Order] If $G$ has a topological order, then $G$ is
    a Dag (prove that there is a contradiction if $G$ has a topo order
    and $G$ has a cycle). If $G$ is a Dag then $G$ contains a node
    with no incoming edges. Topo sort is $\mathcal{O}(m+n)$.
\begin{verbatim}
queue R
graph G
R = topo(G)

queue topo(G*,R*)
    if (G.size() > 1)
        Find a node v with no incoming edges
        R->append(v)
        G->delete(v);
        topo(G,R);
     else
        return(R);
end
\end{verbatim}
  \item[Interval Scheduling] Consists of Earliest Start Time, Shortest
    Interval, Fewest Conflicts, and Earliest Finish Time
  \item[Greedy Algorithm] In general, problems that can be solved by a
    greedy approach exhibit: optimal substructure -- if an optimal
    solution to a problem contains within it optimal solutions to
    subproblems (many); greedy-choice property -- a globally optimal
    solution can be arrived at by making a locally optimal choice
    (one). The greedy algo stays ahead: After each step, its solution
    is at least as good as any others'. Exchange argument: gradually
    transofrm any solution to the one foudn by greedy without hurting
    its quality. Structural: Discover a simple ``structural'' bound
    asserting that every possible optimal solution must have a certain
    value. Then show that your algo achieves this bound.
\begin{verbatim}
Sort jobs by inish times such that f_0 <= f_1 <= f_2, etc
A <- phi
    for (j = 0; j<n; ++j)
        if (job j compatible with A)
            A <- union(A, {j})
        end
    end
    return A
end
\end{verbatim}
In task scheduling, the greedy algorithm is optimal for task-scheduling.
\item[Minimum Spanning Trees] Connected undirected graph $G=(V,E)$
  where each edge $(u,v) \in E$ associated with weight $w(u,v)$
  specifying the cost of the edge. Find an acyclic subset $T
  \subseteq E$ that connects all the vertices in $V$ and whose total
  weight is minimized. Greedy algorithms to create min spanning trees
  make the ``best'' choice at the moment but are not always optimal.
\begin{verbatim}
A <- null
while (!is_mst(A))
    find an edge (u,v) that is "safe" for A
    A <- union(A, {u,v})
end
return A
\end{verbatim}
\item[Finding a safe edge] Let $G = (V, E)$ be a connected undirected
  graph with a real-valued weight function w defined on $E$. Let $A$
  be a subset of $E$ that is included in some minimum spanning tree
  for $G$, let $(S, V − S)$ be any cut of $G$ that respects $A$, and
  let $(u, v)$ be a light edge crossing $(S, V − S)$. Then $(u, v)$ is
  a safe edge for $A$.
\item[Cut] A cut $(S,V-S)$ of an undirected graph is a partition of
  $V$. An edge $(u,v)$ crosses a cut if one of $u$ and $v$ is in $S$
  and the other is in $V-S$. A cut respects a set $A$ of edges if no
  edges in $A$ cross the cut. An edge is a light edge crossing a cut
  if its weight is the minimum of all edges that cross the cut.
\item[Divide \& Conquer] Divide the problem into subproblems. Conquer
  the subproblems by solving them recursively or directly. Combine the
  solutions to the subproblem for the original problem.
  \end{description} % multicols
\end{multicols}
\end{document}