\documentclass[8pt]{article}

\usepackage{mathtools}
\usepackage{amsfonts}
\usepackage{tabularx}
\usepackage{mathabx}
\usepackage{enumitem}
\usepackage{caption}
\usepackage{multicol}
\usepackage{sectsty}
\usepackage{extsizes}
\usepackage{fancyhdr}
\usepackage{enumerate}
\usepackage[margin=1.5cm]{geometry}
\usepackage{multirow}

\pagestyle{fancyplain}

% Try very hard not to break relational and binary operators (equations)
\relpenalty=9999
\binoppenalty=9999

\newcommand{\dd}[1]{\mathrm{d}{#1}}
\newcommand{\ddt}[1]{\frac{\dd{}}{\dd{#1}}}
\newcommand{\dddt}[1]{\frac{\dd{}^2}{\dd{#1}^2}}

\begin{document}

\lhead{Hershal Bhave \quad\tiny{hb6279}}
\rhead{EE360C Fall 2014 Final Cheat Sheet (Pedro Santacruz)}

\begin{multicols}{2}
  \begin{description}
  \item [Flow Network] Directed graph with no parallel edges, with a
    source routing to a sink; each edge has a capacity.
    The net flow sent across the cut is equal to the amount leaving
    $s$:
    $$\sum_{e \text{ leaving } A}f(e) - \sum_{e \text{ entering }
      A}f(e) = v(f)$$
    The value of a flow $f$ is $v(f) = \sum_{e \text{ leaving } s}f(e)$.
    The frontier cut capacity must never exceed the value of the flow
    (Weak Duality Thm).
    If $v(f) = \text{cap}(A,B)$ then $f$ is a max flow and $(A,B)$ is
    a min cut.
  \item [S-T Cut] A partition $(A, B)$ of $V$ with $s \in A$ and $t
    \in B$. The capacity of the cut is $\verb|cap|(A,B)=\sum_{e \text{
      leaving } A}c(e)$.
  \item [S-T Flow] Function which satisfies
    \begin{description}
    \item [Capacity] $\forall e \in E: 0 \le f(e) \le c(e)$
    \item [Conservation] $\forall v \in V- \{s,t\}: \sum_{e\text{ entering
      } v}f(e) = \sum_{e \text{ leaving } v} f(e)$
    \end{description}
  \item [Max Flow Algorithm] Greedy attempts to locally optimize but
    does not always achieve global optimality.
  \item [Max Flow Min Cut (Ford-Faulkerson)] Augment the path by
    adding backward feedback for the flow consumed through a path. If
    there are no more augmenting paths then the flow is maximized
    (Augmenting Path Thm). The value of the max flow is equal to the
    value of the min cut (Max-Flow Min-Cut Thm). Prove these two by
    proving that the following is equivalent:
    \begin{enumerate}
      \item $\exists \text{cut} (A,B) \text{s.t.} v(f) = \text{cap}(A,B)$
      \item Flow $f$ is a max flow
      \item There is no augmenting path relative to $f$.
    \end{enumerate}
  \end{description} % multicols
\end{multicols}
\end{document}
