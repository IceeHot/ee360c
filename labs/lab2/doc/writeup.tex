\documentclass{article}
\title{EE360C Programming Homework 2}
\author{Hershal Bhave (hb6279)}
\date{Due October 28, 2014}

\begin{document}
\maketitle

\section{Proof}
The basic problem for this assignment can be modeled through a DFS
search through a weighted undirected graph. Each edge is disabled once
it has been ``touched'' so that it is never visited again. The queue
of nodes is maintained by a common \verb|std::vector| so that when an
edge is being used it calls \verb|push_back| and when the recursive
call falls through it calls \verb|pop_back|. When the desired node is
found, the stack is not modified and the base case stops the
recursion. This ensures that only the working solution is being
preserved in the \verb|std::vector|.

\section{Pseudo-Code}
{  \footnotesize
\begin{verbatim}
auto undirected_graph::dfs(graph_query* query,
                           size_t curr_dev_id,
                           size_t curr_edge_weight) -> bool {

    if (curr_dev_id == query->get_device_j()) {
        /* Found */
        std::cout << "FOUND: " << curr_dev_id << "\n";
        return true;
    } else {
        auto adj_nodes = nodes[curr_dev_id]->get_adjacent_nodes();

        for (const auto an : adj_nodes) {
            if ((an->nref->is_enabled(curr_dev_id)) &&
                (curr_edge_weight >= query->get_time_a()) &&
                (an->edge_weight >= curr_edge_weight) &&
                (an->edge_weight <= query->get_time_b())) {

                /* Disable the connection */
                an->nref->disable(curr_dev_id);
                nodes[curr_dev_id]->disable(an->nref->get_id());

                query->push_trace(curr_dev_id, an->nref->get_id(), an->edge_weight);
                std::cout << "    pushing: i: " << curr_dev_id
                          << " j: " << an->nref->get_id()
                          << " w: " << an->edge_weight << "\n";

                if (dfs(query, an->nref->get_id(), an->edge_weight)) {
                    return true;
                } else {
                    query->pop_trace();
                }

                /* Not found, then backtrack */
                std::cout << "    backtracking: " << curr_dev_id << "\n";
            }
        }
    }
    return false;
}
\end{verbatim}
}

\section{Runtime Complexity}
Since this algorithm uses a pure DFS approach with no other dependent
in-loop algorithms, the runtime is $\mathcal{O}(n+m)$.

\end{document}